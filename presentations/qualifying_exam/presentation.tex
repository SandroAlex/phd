\documentclass[11pt]{beamer}

\mode<presentation> {
  \usetheme{Singapore} 
  \usecolortheme{seagull}
  \setbeamertemplate{section in toc}[ball unnumbered]
  \setbeamerfont{section in toc}{size=\scriptsize}
  \setbeamertemplate{subsection in toc}[ball unnumbered]
  \setbeamerfont{subsection in toc}{size=\scriptsize}
  \usebeamercolor[fg]{page number in head/foot}
  \setbeamertemplate{footline}[page number]
  \setbeamerfont{page number in head/foot}{size=\large}
  \setbeamercolor{page number in head/foot}{fg=black}
  \setbeamercolor{block title}{bg=blue!50, fg=black}
  \setbeamertemplate{navigation symbols}{}
}

\usepackage{graphicx}
\usepackage{booktabs}
\usepackage{newverbs}
\usepackage[utf8]{inputenc}
\usepackage[brazil]{babel}
\usepackage{amsmath}
\usepackage[style=authoryear]{biblatex}
\usepackage{xcolor}
\usepackage{anyfontsize}
\usepackage{amsfonts}
\usepackage{listings}
\usepackage{amssymb}
\usepackage{url}
\usepackage[version=3]{mhchem}
\usepackage{caption}
\usepackage{multirow}
\usepackage{amsthm}
\usepackage{pythonhighlight}
\usepackage{hyperref}
\usepackage{tikz}
\usepackage{bm}
\usetikzlibrary{arrows, automata}
\graphicspath{
  {/home/alex/Dropbox/doutorado/figuras/}
  {/home/alex/Dropbox/repositories/phd/images/}
}

\hypersetup{
  colorlinks=true,
  linkcolor=black,
  filecolor=cyan,      
  urlcolor=magenta,
}

\newcommand{\btVFill}{\vskip0pt plus 1filll}

\usepackage{ragged2e}
\justifying
\renewcommand{\raggedright}{\leftskip=0pt \rightskip=0pt plus 0cm}
\addtobeamertemplate{block begin}{}{\justifying\setlength{\parindent}{2em}}

\newverbcommand{\bverb}{\color{blue}}{}
\newverbcommand{\gverb}{\color{green}}{}
\newverbcommand{\kverb}{\color{black}}{}
\newverbcommand{\rverb}{\color{red}}{}

\makeatletter
\newlength\beamerleftmargin
\setlength\beamerleftmargin{\Gm@lmargin}
\makeatother

% João (1988)
\newcommand{\citatu}[2]{\textbf{\textcolor{blue}{#1 (#2)}}}        

% (João, 1988)
\newcommand{\citapu}[2]{\textbf{\textcolor{blue}{(#1, #2)}}}       

 % João et al. (1988)
\newcommand{\citat}[2]{\textbf{\textcolor{blue}{#1 et al. (#2)}}}

% (João et al., 1988)
\newcommand{\citap}[2]{\textbf{\textcolor{blue}{(#1 et al., #2)}}} 

 % João et al., 1988 [Para citar varios]
\newcommand{\citav}[2]{\textbf{\textcolor{blue}{#1 et al., #2}}}

 % João, 1988 [Para citar varios]
\newcommand{\citavu}[2]{\textbf{\textcolor{blue}{#1, #2}}}

%%%%%%%%%%%%%%%%%%%%%%%%%%%%%%%%%%%%%%%%%%%%%%%%%%%%%%%%%%%%%%%%%%%%%%%%%%%%%%%%
\begin{document}

\title[
  \textbf{\textcolor{blue}{
      Presentation
  }} 
]{\\[0.01cm]
  \large{\textbf{\textcolor{black}{
        Qualifying Exam
  }}} \\
  
  \small{\textcolor{black}{
      \emph{Presentation}
  }} \\[0.50cm]

  \large{\textbf{\textcolor{black}{
        Machine Learning and Causality for Studies of Interactions Between
        Climate and Vegetation in South America 
  }}}
} 

\author{
  \textcolor{black}{\textbf{
      Alex Sandro Alves de Araujo 
  }} \\ \and Supervisor: Prof. Dr. Henrique de Melo Jorge Barbosa
} 

\institute[
  University of São Paulo - Physics Institute
]{
  \textcolor{black}{
    University of São Paulo - Physics Institute \\ 
    \medskip
    \url{alex.araujo@usp.br}
  }
}

\date{
  \textcolor{black}{
    \today
  }
}

%%%%%%%%%% Slide 1 %%%%%%%%%%
{
  \usebackgroundtemplate{
    \centering
    \includegraphics[
      width=\paperwidth, height=0.92\paperheight
    ]{background.jpg}
  }
  \begin{frame}
    \titlepage 
  \end{frame}
}

%%%%%%%%%% Slide 2 %%%%%%%%%%
\begin{frame}
  \frametitle{\normalsize{\textbf{
        Presentation Structure
  }}}
  
  \tableofcontents
  
\end{frame}

%%%%%%%%%% Slide 3 %%%%%%%%%%
\begin{frame}
  \frametitle{\normalsize{\textbf{
        Presentation Structure
  }}}

  %%%%%%%%%%%%%%%%%%%%%%%%%%%%%%%%%%%%%%%%%%%%%%%%%%%%%%%%%%%%%%%%%%%%%%%%%%%%%
  \section{Motivation}
  %%%%%%%%%%%%%%%%%%%%%%%%%%%%%%%%%%%%%%%%%%%%%%%%%%%%%%%%%%%%%%%%%%%%%%%%%%%%%
  \tableofcontents[currentsection] 

\end{frame}

%%%%%%%%%% Slide 4 %%%%%%%%%%
\begin{frame}[fragile]
  \frametitle{\normalsize{\textbf{
        \emph{Why is it Important?}
  }}} 

  \scriptsize{  
    
  \begin{figure}[h!]
    \centering
    \includegraphics[height=0.65\textheight, keepaspectratio]{interactions3.png}
    \caption*{\scriptsize{
        Adapted from \citatu{Nobre}{2004}.
    }}
  \end{figure}   
      
  \textbullet \: Because of the strong two-way relationship between terrestrial
  vegetation and climate variability, \textbf{predictions of future climate
    change can be improved} through a better understanding of vegetation's
  response to past climate variability.
  }

\end{frame}

%%%%%%%%%% Slide 5 %%%%%%%%%%
\begin{frame}[fragile]
  \frametitle{\normalsize{\textbf{
        \emph{Land, Atmosphere, and Radiation}
  }}} 

  \scriptsize{
    
    \textbullet \: Their interactions are not all well understood and are often
    poorly represented in numerical models:
    \begin{enumerate}
    \item Interactions occur at different spatial and temporal scales.
    \item They can be highly nonlinear, with complex feedbacks.
    \item Clearly dependence on the geographical location.
    \end{enumerate}

  }
  
  \begin{figure}[h!]
    \centering
    \includegraphics[height=0.625\textheight]{interactions2.png}
    \caption*{\scriptsize{
        \citat{Dirmeyer}{2019}
    }}
  \end{figure}

\end{frame}

%%%%%%%%%% Slide 6 %%%%%%%%%%
\begin{frame}[fragile]
  \frametitle{\normalsize{\textbf{
          South America \\
          Land Cover and Köppen-Geiger Climate Classification
  }}} 

  \begin{figure}[h!]
    \centering
    \includegraphics[width=1.075\textwidth, keepaspectratio]
                    {south_america_koppen_geiger_vegetation.png}
    \caption*{\scriptsize{
        \textbf{(A):} IGBP land cover classification, 1992-1993
        \citap{Loveland}{2009} \textbf{(B):} Types of climates \citap{Beck}{2018}. 
    }}
  \end{figure}
   
\end{frame}

%%%%%%%%%% Slide 7 %%%%%%%%%%
\begin{frame}[fragile]
  \frametitle{\normalsize{\textbf{
        \emph{Increase in Volume and Complexity}
  }}} 

  \vspace{0.25cm}
  
  \scriptsize{

    \textbullet \: The volume of worldwide \textbf{climate data} is expanding
    rapidly!
    
  \begin{figure}[h!]
    \centering
    \includegraphics[width=1.05\textwidth]{climate_data.png}
    \caption*{\scriptsize{
        \citat{Overpeck}{2011}
      }}
  \end{figure}
    
  }
\end{frame}

%%%%%%%%%% Slide 8 %%%%%%%%%%
\begin{frame}[fragile]
  \frametitle{\normalsize{\textbf{
        \emph{Big Data Techniques in Earth Science}
  }}} 

  \vspace{0.25cm}
  
  \scriptsize{

    \textbullet \: Can be useful for climate research if combined with more
    traditional approaches based on domain-specific knowledge \citap{Knüsel}{2019}.
    
    \begin{figure}[h!]
      \centering
      \includegraphics[height=0.725\textheight]{big_data_earth.png}
      \caption*{\scriptsize{
          \citat{Reichstein}{2019}
      }}
    \end{figure}
        
  }
  
\end{frame}

%%%%%%%%%% Slide 9 %%%%%%%%%%
\begin{frame}[fragile]
  \frametitle{\normalsize{\textbf{
        \emph{Two Modeling Approaches}
  }}} 

  \scriptsize{

    \textbullet \: Process based \textbf{(A):} \: Described by equations. Usually
    without analytical solutions. Sophisticated numerical
    methods. \textcolor{red}{Many parameters to be set}. \\[0.10cm]

    \textbullet \:Data Based \textbf{(B):} \: Search for statistical relations
    between input and output data. It does not care about underlying
    process. \textcolor{red}{Data hungry, reproducibility, conservation
      laws}. \\[0.10cm]

    \textbullet \: Can we combine both? Yes, we must do it.
    
    \begin{figure}[h!]
      \centering
      \includegraphics[width=0.90\textwidth]{numerical_and_data_driven_models.png}
      \caption*{\scriptsize{
          \href{https://medium.com/@b.bhaskaran/data-driven-statistical-models-vs-process-driven-physical-models-340f4dd4eea8}{Source
            here}
      }}
    \end{figure}
    
  }
  
\end{frame}

%%%%%%%%%% Slide 10 %%%%%%%%%%
\begin{frame}[fragile]
  \frametitle{\normalsize{\textbf{
        \emph{What is possible to do with machine (deep) learning?}
  }}} 

  \begin{figure}[h!]
    \centering
    \includegraphics[width=\textwidth]{applications_dl3.png} \\
    \includegraphics[width=\textwidth]{applications_dl4.png}
    \caption*{\scriptsize{
        \citat{Reichstein}{2019} 
      }}
  \end{figure}
  
\end{frame}

%%%%%%%%%% Slide 11 %%%%%%%%%%
\begin{frame}[fragile]
  \frametitle{\normalsize{\textbf{
        \emph{What is possible to do with causal inference?}
  }}} 

  \begin{figure}[h!]
    \centering
    \includegraphics[height=0.85\textheight]{applications_causal.png}
    \caption*{\scriptsize{
        \citat{Runge}{2019a} 
      }}
  \end{figure}
  
\end{frame}

%%%%%%%%%% Slide 12 %%%%%%%%%%
\begin{frame}[fragile]
  \frametitle{\normalsize{\textbf{
        \emph{Data-Driven Challenges in Earth System Sciences}
  }}} 

  \scriptsize{

    \vspace{0.25cm}
    
    \textbullet \: Machine learning: \textcolor{red}{\textbf{(1)}
    Interpretability}, \textbf{(2)} Physical consistency, \textbf{(3)} Complex
    and uncertain data, \textbf{(4)} Limited labels, and
    \textcolor{red}{\textbf{(5)} Computational demand}. \\[0.10cm] 
    
    \textbullet \: Causal discovery:
    
    \begin{figure}[h!]
      \centering
      \includegraphics[height=0.68\textheight]{applications_causal2.png}
      \caption*{\scriptsize{
          Adapted from \citat{Runge}{2019a}.
      }}
    \end{figure}
    
  }
  
\end{frame}

%%%%%%%%%% Slide 13 %%%%%%%%%%
\begin{frame}
  \frametitle{\normalsize{\textbf{
        Presentation Structure
  }}}

  %%%%%%%%%%%%%%%%%%%%%%%%%%%%%%%%%%%%%%%%%%%%%%%%%%%%%%%%%%%%%%%%%%%%%%%%%%%%%
  \section{Objectives}
  %%%%%%%%%%%%%%%%%%%%%%%%%%%%%%%%%%%%%%%%%%%%%%%%%%%%%%%%%%%%%%%%%%%%%%%%%%%%%
  \tableofcontents[currentsection] 
  
\end{frame}

%%%%%%%%%% Slide 14 %%%%%%%%%%
\begin{frame}
  \frametitle{\normalsize{\textbf{
        Scope and Objectives
  }}}

  \scriptsize{
    
    \textbullet \: General goals of this work:

    \begin{enumerate}
      
    \item A better understanding of the interactions between climate and
      vegetation in South America, specially in the Amazon rain forest. 
      
    \item Do that using relatively large amount of data and analysis tools based
      on machine learning and causal inference methods.
      
    \end{enumerate}
      
    \textbullet \: Specific goals at current stage:

    \begin{enumerate}

    \item Extensive spatial and temporal characterization of vegetation data.

    \item Initial studies for interactions between climate and vegetation.
      
    \item Moisture transport and its trajectories.
      
    \end{enumerate}
    
  }
  
\end{frame}

%%%%%%%%%% Slide 15 %%%%%%%%%%
\begin{frame}
  \frametitle{\normalsize{\textbf{
        Presentation Structure
  }}}

  %%%%%%%%%%%%%%%%%%%%%%%%%%%%%%%%%%%%%%%%%%%%%%%%%%%%%%%%%%%%%%%%%%%%%%%%%%%%%
  \section{Methodology}
  %%%%%%%%%%%%%%%%%%%%%%%%%%%%%%%%%%%%%%%%%%%%%%%%%%%%%%%%%%%%%%%%%%%%%%%%%%%%%
  \tableofcontents[currentsection] 
  
\end{frame}

%%%%%%%%%%%%%%%%%%%%%%%%%%%%%%%%%%%%%%%%%%%%%%%%%%%%%%%%%%%%%%%%%%%%%%%%%%%%%%%
\subsection{Data: Sources, Hypercubes, and Tools}
%%%%%%%%%%%%%%%%%%%%%%%%%%%%%%%%%%%%%%%%%%%%%%%%%%%%%%%%%%%%%%%%%%%%%%%%%%%%%%%

%%%%%%%%%% Slide 16 %%%%%%%%%%
\begin{frame}
  \frametitle{\normalsize{\textbf{
        Data Source for Vegetation (GIMMS NDVI3g 1981 - 2015)
  }}} 

  \scriptsize{
    
    \begin{columns}
      \column{0.50\linewidth}

      \textbullet \: Vegetation indices are radiometric
      measures of photosynthetically active radiation absorbed by chlorophyll in
      plants. \\

      \textbullet \:They are surrogates of vegetation development as well as
      land use change. \\

      \textbullet \: \textbf{N}ormalized \textbf{D}ifference \textbf{V}egetation
      \textbf{I}ndex NDVI. \\[0.10cm]  
      
      \textbullet \: Plants absorb radiation at red, but they reflect at near
      infrared in order to avoid excessive heating of its internal tissues:

      \begin{equation}
        \boxed{\text{NDVI} = \frac{\rho_{\text{infra}} -
          \rho_{\text{red}}}{\rho_{\text{infra}} + \rho_{\text{red}}} \nonumber}
      \end{equation} 
      where $\rho_{\text{infra}}$ are $\rho_{\text{red}}$ are reflectances. 
      
      \column{0.60\linewidth}

      \begin{figure}[h!]
        \centering
        \includegraphics[width=\linewidth, keepaspectratio]{ndvi_gimms.png}
        \caption*{\scriptsize{
            \citat{Pinzon and Tucker}{2014}
        }}
      \end{figure}
      
    \end{columns}

    \vspace{0.10cm}
    
    \textbf{\textbullet \: Metadata:} (1) netCDF format; (2) Global domain;
    \textcolor{red}{(3) 1/12 degree lat-2160 lon-4320}; (4) Daily time step;
    \textcolor{red}{(5) Biweekly from 1981/07 to 2015/12}; (6) AVHRR input data. 
   
  }
      
\end{frame}

%%%%%%%%%% Slide 17 %%%%%%%%%%
\begin{frame}
  \frametitle{\normalsize{\textbf{
        Data Sources for Climate (ERA-Interim 1979 - 2019)
  }}}

  \scriptsize{
    
    \begin{columns}
      \column{0.55\linewidth}

      \begin{figure}[h!]
        \centering
        \includegraphics[width=0.88\linewidth, keepaspectratio]{erainterim3.png}\\
        \includegraphics[width=0.88\linewidth, keepaspectratio]{erainterim4.png}
        \caption*{\scriptsize{
            \citat{Dee}{2011}  
        }}
      \end{figure}
      
      \column{0.55\linewidth}
      
      \textbullet \: Reanalyses combine past observations with models to
      generate consistent time series of multiple climate
      variables. \\
      
      \textbf{\textbullet \: Selected features:} (1) Total precipitation; (2)
      Surface temperature; (3) Surface net solar radiation; (4) Soil moisture;
      (5) Soil temperature; (6) Evaporation. \\

      \textbf{\textbullet \: New features:} (7) Surface relative humidity; (8)
      Water balance; (9) Water deficit. \\

      \textbf{\textbullet \: Metadata:} (1) netCDF and GRIB formats; (2) Global
      Domain; (3) \textcolor{red}{Spatial resolution 0.75 x 0.75 degrees} 37
      levels; (4) \textcolor{red}{Monthly} time steps; (5) From 1979/01 to
      2019/12. 
      
    \end{columns}
  }  
  
\end{frame}

%%%%%%%%%% Slide 18 %%%%%%%%%%
\begin{frame}
  \frametitle{\normalsize{\textbf{
        Weather and Climate Data
  }}}

  \scriptsize{
    
    \begin{figure}[h!]
      \centering
      \includegraphics[width=\textwidth, height=\textheight, keepaspectratio]
                      {hypercubes.png} \\[0.35cm]
      \includegraphics[height=0.275\textheight, keepaspectratio]{table_ml.png}
  \end{figure}
    
  }
  
\end{frame}

%%%%%%%%%% Slide 19 %%%%%%%%%%
\begin{frame}
  \frametitle{\normalsize{\textbf{
        Tools for Climate Science
  }}} 

  \scriptsize{
    
    \begin{figure}[h!]
      \centering
      \includegraphics[width=\textwidth, height=\textheight,keepaspectratio]
                      {tools.jpg}
    \end{figure}
    
  }
\end{frame}

%%%%%%%%%%%%%%%%%%%%%%%%%%%%%%%%%%%%%%%%%%%%%%%%%%%%%%%%%%%%%%%%%%%%%%%%%%%%%%%
\subsection{Statistical Learning Models}
%%%%%%%%%%%%%%%%%%%%%%%%%%%%%%%%%%%%%%%%%%%%%%%%%%%%%%%%%%%%%%%%%%%%%%%%%%%%%%%

%%%%%%%%%% Slide 20 %%%%%%%%%%
\begin{frame}
  \frametitle{\normalsize{\textbf{
        Machine Learning
  }}} 

  \scriptsize{
    
    \textbullet \: \textbf{Tom Mitchell (1997):}

    \begin{quote}
      A computer program is said to learn if its performance at a task T, as
      measured by a performance P, improves with experience E.
    \end{quote}

    \begin{columns}
      \column{0.55\linewidth}
      
      \begin{figure}[h!]
        \centering
        \includegraphics[width=\linewidth, keepaspectratio]{ml101.png}
        \caption*{\tiny{
            \citat{Chollet}{2018}
        }}
      \end{figure}
      
      \column{0.55\linewidth}
      
      \textbf{Unsupervised learning}:
      \begin{enumerate}
      \item Clusterization and dimensionality reduction.
      \item Data without input/output.
      \item Find patterns.
      \item \textcolor{red}{EOFs or PCA and K-Means}.
      \end{enumerate}

      \textbf{Supervised learning}:
      \begin{enumerate}
      \item Classification and regression.
      \item Input/output given.
      \item Predictions.
      \item \textcolor{red}{Random forests and neural networks}.
      \end{enumerate}
      
    \end{columns}
      
  }
  
\end{frame}

%%%%%%%%%% Slide 21 %%%%%%%%%%
\begin{frame}
  \frametitle{\normalsize{\textbf{
        Random Forests
  }}} 

  \scriptsize{
    
    \begin{columns}
      \column{0.55\linewidth}

      \textbullet \: Random \textbf{forests} are made of decision
      \textbf{trees}. \\
      
      \textbullet \: Classification and regression. 

      \begin{figure}[h!]
        \centering
        \includegraphics[width=\linewidth, keepaspectratio]{decision_tree.png}
        \caption*{\scriptsize{
            Iris decision tree \citap{Géron}{2019}.
        }}
      \end{figure}
      
      \column{0.55\linewidth}

      \textbullet \: Independent predictors with subsets of features and
      bootstrapping.  \\

      \textbullet \: Majority vote for classification or mean target for regression.
      
      \begin{figure}[h!]
        \centering
        \includegraphics[width=0.9\linewidth, keepaspectratio]{rf.png}\\
        \includegraphics[width=\linewidth, keepaspectratio]{rf2.png}
        \caption*{\scriptsize{
            \citat{Géron}{2019}
        }}
      \end{figure}
      
    \end{columns}
  }  
  
\end{frame}

%%%%%%%%%% Slide 22 %%%%%%%%%%
\begin{frame}
  \frametitle{\normalsize{\textbf{
        Neural Networks
  }}} 

  \scriptsize{
    
    \begin{columns}
      \column{0.55\linewidth}

      \textbullet \: Brain’s architecture for inspiration on how to build an
      intelligent machine. \\
      
      \textbullet \: Fundamental processing units are neurons.  
      
      \begin{figure}[h!]
        \centering
        \includegraphics[width=0.8\linewidth,keepaspectratio]
                        {neural_network_basics.png}
        \caption*{\scriptsize{
            Basic architecture parts of a simple artificial neural
            network \citap{Mehta}{2019}. 
        }}
      \end{figure}
      
      \column{0.55\linewidth}
      
      \textbullet \: Recurrent neural networks (RNNs) are used for sequential
      data, but there may be some problems.   
      
      \begin{figure}[h!]
        \centering
        \includegraphics[width=0.75\linewidth, keepaspectratio]{recurrent2.png}
        \caption*{\scriptsize{
            Idea behind RNNs \citapu{Olah}{2015}. 
        }}
      \end{figure}

      
      \textbullet \: Long Short Term Memory networks (LSTMs) solve these
      problems because they can manage context. 
      
      \begin{figure}[h!]
        \centering
        \includegraphics[width=0.75\linewidth, keepaspectratio]{lstm_cell.png}
        \caption*{\scriptsize{
            LSTM unit \citapu{Olah}{2015}.
        }}
      \end{figure}
      
    \end{columns}
  }    
  
\end{frame}

%%%%%%%%%% Slide 23 %%%%%%%%%%
\begin{frame}
  \frametitle{\normalsize{\textbf{
        Supervised Learning on Time Series
  }}} 

  \scriptsize{

    \begin{columns}
      \column{0.55\linewidth}

      \vspace{0.10cm}
      
      \textbullet \: 414 or 828 data instances at each grid point. \\
      
      \textbullet \: Cross-validation is a resampling procedure used to evaluate
      machine learning models on a \textbf{limited} data sample.
       
      \begin{figure}[h!]
        \centering \includegraphics[width=0.95\linewidth,
          keepaspectratio]{kfolds.png}
        \caption*{\scriptsize{
            \href{https://scikit-learn.org/stable/modules/cross_validation.html}{scikit-learn
              page}  }}
      \end{figure}
            
      \column{0.55\linewidth}

      \textbullet \: Time series data require special attention for training
      machine learning models: (1) Auto-correlation; (2) Stationarity; (3)
      Look-ahead bias.
      
      \begin{figure}[h!]
        \centering
        \includegraphics[width=\linewidth, keepaspectratio]{kfolds2.png}
        \caption*{\scriptsize{
            \href{https://towardsdatascience.com/time-series-nested-cross-validation-76adba623eb9}{towards
              data science}
        }}
      \end{figure}
      
    \end{columns}
    
  }
      
\end{frame}

%%%%%%%%%% Slide 24 %%%%%%%%%%
\begin{frame}
  \frametitle{\normalsize{\textbf{
        Time Series Decomposition
  }}} 
    
  \begin{figure}[h!]
    \centering
    \includegraphics[height=0.925\textheight, keepaspectratio]
                    {ndvi_gimms_anomaly_calculation_pt.jpg} 
  \end{figure}
  
\end{frame}

%%%%%%%%%% Slide 25 %%%%%%%%%%
\begin{frame}
  \frametitle{\normalsize{\textbf{
        Empirical Orthogonal Functions
  }}} 

  \scriptsize{
    
    \begin{columns}
      \column{0.55\linewidth}

      \textbullet \: Method of finding structures (modes) that explain maximum
      variance in two-dimensional data sets (space-time).
      
      \begin{figure}[h!]
        \centering
        \includegraphics[width=0.625\linewidth, keepaspectratio]{eof.jpg}
      \end{figure}
      
      \textbullet \: \textbf{Orthogonal Modes}: Descending order in variance
      explanation. 
      
      \textbullet \: \textbf{EOFs} are spatial patterns and \textbf{PCs} are
      time behavior of these patterns. 

      \textbullet \: \textcolor{red}{Not necessarily physical modes!}
      $$
      \vec{F}(\vec{x}, t) = \sum_{k=1}^{N} PC_{k}(t) \cdot \vec{EOF}_k(\vec{x})
      $$
      
      \column{0.55\linewidth}

      \begin{figure}[h!]
        \centering
        \includegraphics[height=0.425\textheight, keepaspectratio]
                        {trmm_eofs.png}\\
        \includegraphics[height=0.40\textheight, keepaspectratio]
                        {trmm_eofs2.png}        
        \caption*{\scriptsize{
            Precipitation 2-EOFs.
        }}
      \end{figure}
      
    \end{columns}
  }    
  
\end{frame}

%%%%%%%%%% Slide 26 %%%%%%%%%%
\begin{frame}
  \frametitle{\normalsize{\textbf{
        K-Means Clustering Algorithm
  }}} 

  \scriptsize{
  
  \textbullet \: Cost function:
  $$
  C = \sum\limits_{k=1}^{K}\sum\limits_{n=1}^{N}r_{nk} || \bm{x}_n - \bm{\mu}_k ||^2
  $$

  \textbullet \: Two-step algorithm:
  $$
  \text{Expectation: \:} \bm{\mu}_k = \frac{1}{N_k} \sum\limits_{n=1}^{N_k}
  r_{nk} \bm{x}_n \hspace{0.25cm} \text{Maximization: \:} 
  r_{nk} = \begin{cases}
    1, & \text{if } k = \text{arg min}_{k'} || \bm{x}_n - \bm{\mu}_{k'} ||^2 \\
    0, & \text{otherwise}
  \end{cases}
  $$
  
  \begin{figure}[h!]
    \centering
    \includegraphics[height=0.45\textheight, keepaspectratio]{kmeans.png} 
    \caption*{\scriptsize{
        \href{https://github.com/MihailaDumitru/K-Means_Clustering/blob/master/README.md}{What
        K-Means does for you.}
    }}
  \end{figure}  

}
  
\end{frame}

%%%%%%%%%%%%%%%%%%%%%%%%%%%%%%%%%%%%%%%%%%%%%%%%%%%%%%%%%%%%%%%%%%%%%%%%%%%%%%%
\subsection{Causality and Proposed Model}
%%%%%%%%%%%%%%%%%%%%%%%%%%%%%%%%%%%%%%%%%%%%%%%%%%%%%%%%%%%%%%%%%%%%%%%%%%%%%%%

%%%%%%%%%% Slide 27 %%%%%%%%%%
\begin{frame}
  \frametitle{\normalsize{\textbf{
        Causal Sufficiency
  }}} 

  \vspace{0.25cm}
  
  \scriptsize{
    
    \textbullet \: A direct causal relation becomes indirect when an additional
    variable is included \citapu{Decubber}{2017}. 
    
    \begin{figure}[!h]
      \centering
      \begin{tikzpicture}[
          > = stealth, % arrow head style
          shorten > = 1pt, % don't touch arrow head to node
          auto,
          align = center,
          text width = 1.25cm, 
          node distance = 2cm, % distance between nodes
          semithick % line style
        ]
    
        \tikzstyle{every state}=[
          draw = black,
          thick,
          fill = white,
          minimum size = 1cm
        ]
        
        \node[state, double=red](v1){\tiny{\textbf{Cloud cover} \\ (y/n)}};
        \node[state, double=red](v2)[right of=v1]{\tiny{\textbf{Flood} \\ (y/n)}};
        \path[->](v1) edge node {} (v2);

        \begin{scope}[xshift=5cm]
          
          \node[state](A){\tiny{\textbf{Cloud cover} \\ (y/n)}};
          \node[state](B)[right of=A]{\tiny{\textbf{Rain} \\ (y/n)}};
          \node[state](C)[right of=B]{\tiny{\textbf{Flood} \\ (y/n)}};
          \path[->](A) edge node {} (B);
          \path[->](B) edge node {} (C);
          
        \end{scope}
        
      \end{tikzpicture}

    \end{figure}

    \textbullet \: Causal relations disappear when a hidden common cause is
    introduced in the study \citapu{Decubber}{2017}.
    
    \begin{figure}[!h]
      \centering
      \begin{tikzpicture}[
          > = stealth, % arrow head style
          shorten > = 1pt, % don't touch arrow head to node
          auto,
          align = center,
          text width = 1.25cm, 
          node distance = 2.10cm, % distance between nodes
          semithick % line style
        ]
    
        \tikzstyle{every state}=[
          draw = black,
          thick,
          fill = white,
          minimum size = 1cm
        ]
        
        \node[state, double=red](A){\tiny{\textbf{Rain} \\ (y/n)}};
        \node[state, double=red](B)[right of=v1]{\tiny{\textbf{UV} \\ (y/n)}};
        \path[->](A) edge node {} (B);
        \path[->](B) edge node {} (A);
        
        \begin{scope}[xshift=5cm]
          
          \node[state](v1){\tiny{\textbf{Cloud cover} \\ (y/n)}};
          \node[state](v2)[below left of=v1]{\tiny{\textbf{Rain} \\ (y/n)}};
          \node[state](v3)[below right of=v1]{\tiny{\textbf{UV} \\ (y/n)}};
          \path[->](v1) edge node {} (v2);
          \path[->](v1) edge node {} (v3);
          
        \end{scope}
        
      \end{tikzpicture}
    \end{figure}

  }

\end{frame}

%%%%%%%%%% Slide 28 %%%%%%%%%%
\begin{frame}
  \frametitle{\normalsize{\textbf{
        Linear Granger Causality
  }}}

  \scriptsize{

    \begin{quote}
      For two simultaneously measured signals, if we can predict the first
      signal better by using the past information from the second one than by
      using the information without it, then we call the second signal causal to
      the first one ($X \rightarrow Y$)
    \end{quote}
    
    \textbullet \: Two time series $X_t$ and $Y_t$, ($P$ is the window size):
    $$
    y_t= f_1(\underbrace{y_{t-1}, y_{t-2}, ..., y_{t-P}}_{\text{target past}})
    $$

    $$
    y_t = f_2(\underbrace{y_{t-1}, y_{t-2}, ..., y_{t-P}}_{\text{target past}},
    \underbrace{x_{t-1}, x_{t-2}, ..., x_{t-P}}_{\text{aux past}})
    $$

    \textbullet \: Three or more time series $X_t$, $Y_t$, $W_t$, ...:
    $$
    y_t= f_1(\underbrace{y_{t-1}, y_{t-2}, ..., y_{t-P}}_{\text{target past}},
    \underbrace{w_{t-1}, w_{t-2}, ..., w_{t-P}}_{\text{\textcolor{red}{confounding}}})
    $$

    $$
    y_t = f_2(\underbrace{y_{t-1}, y_{t-2}, ..., y_{t-P}}_{\text{target past}},
    \underbrace{x_{t-1}, x_{t-2}, ..., x_{t-P}}_{\text{aux past}},
    \underbrace{w_{t-1}, w_{t-2}, ..., w_{t-P}}_{\text{\textcolor{red}{confounding}}})
    $$    
    
  }
 
\end{frame}

%%%%%%%%%% Slide 29 %%%%%%%%%%
\begin{frame}
  \frametitle{\normalsize{\textbf{
        Nonlinear Extension of Granger Causality
  }}}

  \scriptsize{

    \vspace{0.20cm}
    
    \textbullet \: \citat{Papagiannopoulou}{2017a} present a novel non-linear framework
      consisting of several components, such as \textbf{(1)} data collection
      from various databases, \textbf{(2)} time series decomposition techniques,
      \textbf{(3)} feature construction methods, and \textbf{(4)} predictive
      modelling by means of random forests.
 
    \begin{figure}[h!]
      \centering
      \includegraphics[width=0.75\linewidth, keepaspectratio]{cg.png}
      \caption*{\scriptsize{
          Climatic and environmental factors controlling vegetation dynamics.
      }}
      \end{figure}

    \textbullet \: \textbf{\textcolor{red}{Possible methodological
        improvements:}} \textbf{(1)} Systematic hyperparameter search to avoid
    overfitting. RFs $\rightarrow$ LSTMs; \textbf{(2)} Avoid look-ahead bias;
    \textbf{(3)} Add more variables and avoid duplicated ones; \textbf{(4)}
    Curse of dimensionality; \textbf{(5)} Include stationarity requirements for
    predictors;
    
  }
  
\end{frame}

%%%%%%%%%% Slide 30 %%%%%%%%%%
\begin{frame}
  \frametitle{\normalsize{\textbf{
        Presentation Structure
  }}}

  %%%%%%%%%%%%%%%%%%%%%%%%%%%%%%%%%%%%%%%%%%%%%%%%%%%%%%%%%%%%%%%%%%%%%%%%%%%%%
  \section{Results}
  %%%%%%%%%%%%%%%%%%%%%%%%%%%%%%%%%%%%%%%%%%%%%%%%%%%%%%%%%%%%%%%%%%%%%%%%%%%%%
  \tableofcontents[currentsection] 
  
\end{frame}

%%%%%%%%%%%%%%%%%%%%%%%%%%%%%%%%%%%%%%%%%%%%%%%%%%%%%%%%%%%%%%%%%%%%%%%%%%%%%%%
\subsection{(R1) Normalized Vegetation Difference Index}
%%%%%%%%%%%%%%%%%%%%%%%%%%%%%%%%%%%%%%%%%%%%%%%%%%%%%%%%%%%%%%%%%%%%%%%%%%%%%%%

%%%%%%%%%% Slide 31 %%%%%%%%%%
\begin{frame}
  \frametitle{\normalsize{\textbf{
        (R1) Summary Statistics of Vegetation
  }}}

  \scriptsize{
  
    \begin{figure}[h!]
      \centering
      \includegraphics[width=0.49\linewidth, keepaspectratio]
                      {ndvi_gimms_mean_ori_pt.jpg}
      \includegraphics[width=0.49\linewidth, keepaspectratio]
                      {ndvi_gimms_std_ori_pt.jpg}                    
    \end{figure}  
    
    \textbullet \: \textbf{Highlights:} (1) Predominant greening of Amazon; (2) Eastern
    Rondônia; (3) Arc of Deforestation; (4) Northeast of Brazil.

  }
    
\end{frame}

%%%%%%%%%% Slide 32 %%%%%%%%%%
\begin{frame}
  \frametitle{\normalsize{\textbf{
        (R1) Seasonality of Vegetation and Precipitation
  }}}

  \scriptsize{

    \textbullet \: \textbf{Highlights:} (1) Greenest Amazon in JJA; (2) Brownest
    in DJF; (3) Precipitation does not seem to correlate for most of Amazon; (4)
    Probably driven by radiation.
    
    \begin{figure}[h!]
      \centering
      \includegraphics[width=0.925\linewidth, keepaspectratio]
                      {ndvi_gimms_smeans_pt.jpg}
      \includegraphics[width=0.925\linewidth, keepaspectratio]
                      {precipitation_trmm_smeans_pt.jpg}                       
      \caption*{\scriptsize{
          TRMM 3B42 product, 1988 to 2017 \citap{Huffman}{2007}.
      }}
    \end{figure}
    
  }
  
\end{frame}

%%%%%%%%%% Slide 33 %%%%%%%%%%
\begin{frame}
  \frametitle{\normalsize{\textbf{
        (R1) Linear Trends for Vegetation
  }}}

  \scriptsize{

    \textbullet \: \textbf{Highlights:} (1) South America is becoming greener;
    (2) Most trendy regions does not show pronounced signal; (3) Eastern
    Rondônia; (4) Arc of deforestation.
    
    \begin{columns}
      \column{0.57\linewidth}
      
      \begin{figure}[h!]
        \centering
        \includegraphics[width=\linewidth, keepaspectratio]
                        {ndvi_gimms_trends_ori.jpg}                       
      \end{figure}  
      
      \column{0.57\linewidth}

      \begin{figure}[h!]
        \centering
        \includegraphics[width=0.80\linewidth, keepaspectratio]
                        {ndvi_gimms_trends_ori_hist.jpg}
        \includegraphics[width=\linewidth, keepaspectratio]
                        {ndvi_gimms_trends_ori_tss.jpg}                          
      \end{figure}  
      
    \end{columns}    
  }
  
\end{frame}

%%%%%%%%%% Slide 34 %%%%%%%%%%
\begin{frame}
  \frametitle{\normalsize{\textbf{
        (R1) Empirical Orthogonal Functions for Vegetation
  }}}

  \scriptsize{

    \textbullet \: \textbf{Highlights:} (1) The first two modes explain less
    than half; (2) First spatial pattern looks like to separate radiation and
    water limited areas; (3) Northeast of Brazil (Zona da Mata - Agreste e
    Sertão); (4) Eastern Rondônia; (5) Latitudinal pattern in second mode
    (Capricorn and Equator).
    
    \begin{columns}
      \column{0.55\linewidth}

      \begin{figure}[h!]
        \includegraphics[width=\linewidth, keepaspectratio]
                        {ndvi_gimms_modes_spatial.jpg}                    
      \end{figure}  
      
      \column{0.55\linewidth}

      \begin{figure}[h!]
        \includegraphics[width=\linewidth, keepaspectratio]
                        {ndvi_gimms_modes_temporal.jpg}                    
      \end{figure}        

    \end{columns}

    \begin{figure}[h!]
      \includegraphics[width=0.70\linewidth, keepaspectratio]
                      {ndvi_gimms_modes_variances.jpg}  
    \end{figure}
    
  }
  
\end{frame}

%%%%%%%%%% Slide 35 %%%%%%%%%%
\begin{frame}
  \frametitle{\normalsize{\textbf{
        (R1) Vegetation Clustering by K-Means
  }}}

  \scriptsize{
    
    \textbullet \: \textbf{Highlights:} (1) Elbow curve and silhouettes; (2)
    Have found 9 vegetation clusters; (3) Amazonia is represented by cluster
    numbers 3, 4, 5, and 6; (4) Mean time series for have different
    characteristics for each cluster.
  
    \begin{columns}
      \column{0.55\linewidth}

      \begin{figure}[h!]
        \centering
        \includegraphics[width=0.95\textwidth, keepaspectratio]
                        {ndvi_gimms_clustering_scaled_map.jpg}                      
      \end{figure}
    
      \column{0.55\linewidth}
      
      \begin{figure}[h!]
        \centering
        \includegraphics[width=\textwidth, keepaspectratio]
                        {ndvi_gimms_clustering_scaled_pt.jpg}          
        \includegraphics[width=\textwidth, keepaspectratio]
                        {ndvi_gimms_clustering_scaled_time_series.jpg}
      \end{figure}   
    
  \end{columns}
  
  }
\end{frame}

%%%%%%%%%% Slide 36 %%%%%%%%%%
\begin{frame}
  \frametitle{\normalsize{\textbf{
        (R1) Vegetation Clustering by K-Means
  }}}

  \scriptsize{
    
    \textbullet \: \textbf{Highlights:} (1) Need for counting coincidences of
    these two approaches: 9 clusters x 14 land cover classes. (2) Difficulties:
    regrid categorical data? IGBP is for 1992-1993.
    
    \begin{figure}[h!]
      \centering
      \includegraphics[width=1.05\textwidth, keepaspectratio]
                      {clustering_ndvi_land_cover.png}          
    \end{figure}       
  }
    
\end{frame}

%%%%%%%%%% Slide 37 %%%%%%%%%%
\begin{frame}
  \frametitle{\normalsize{\textbf{
        (R1) Conclusions for Vegetation Analysis
  }}}

  \scriptsize{

    \textbullet \: Amazonia has predominant greening along the year. Deforested
    areas present lower mean values (Rondônia and Arc of deforestation). Greener
    on JJA and browner on DJF. \\[0.15cm]

    \textbullet \: South America is becoming greener, but not so
    much. \\[0.15cm]

    \textbullet \: First EOF seems to separate radiation and water limited
    areas. Second EOF has a latitudinal pattern that we do not
    understood. \\[0.15cm]

    \textbullet \: We have found 9 vegetation clusters. We need to further
    evaluate if these clusters match land cover types. \\[0.15cm]
    
  }
    
\end{frame}

%%%%%%%%%%%%%%%%%%%%%%%%%%%%%%%%%%%%%%%%%%%%%%%%%%%%%%%%%%%%%%%%%%%%%%%%%%%%%%%
\subsection{(R2) Initial Investigations of Climate and Vegetation}
%%%%%%%%%%%%%%%%%%%%%%%%%%%%%%%%%%%%%%%%%%%%%%%%%%%%%%%%%%%%%%%%%%%%%%%%%%%%%%%

%%%%%%%%%% Slide 38 %%%%%%%%%%
\begin{frame}
  \frametitle{\normalsize{\textbf{
        (R2) Observed Climate and Vegetation for Manaus
  }}}
    
  \begin{figure}[h!]
    \centering
    \includegraphics[height=0.85\textheight, keepaspectratio] 
                    {time_series_manaus_pt.jpg} 
    \includegraphics[height=0.85\textheight, keepaspectratio]
                    {pair_plots_manaus_pt.jpg}
    \end{figure}        
    
\end{frame}

%%%%%%%%%% Slide 39 %%%%%%%%%%
\begin{frame}
  \frametitle{\normalsize{\textbf{
        (R2) Observed Climate and Vegetation for Manaus
  }}}
  
  \begin{figure}[h!]
    \centering
    \includegraphics[width=0.75\textwidth, keepaspectratio]
                    {time_series_manaus_pt_zoom.jpg} \\
    \includegraphics[width=\textwidth, keepaspectratio]
                    {pair_plots_manaus_pt_zoom.jpg}                 
  \end{figure}        
    
\end{frame}


%%%%%%%%%% Slide 40 %%%%%%%%%%
\begin{frame}
  \frametitle{\normalsize{\textbf{
        (R2) Spearman's Rank Correlation Coefficient
  }}} 

  \begin{columns}
    \column{0.55\linewidth}  
  
    \begin{figure}[h!]
      \centering
      \includegraphics[height=0.85\textheight, keepaspectratio]
                        {ndvi_climate_spearman_pt.jpg}                    
    \end{figure}   

    \column{0.55\linewidth}

    \begin{figure}[h!]
      \centering
      \includegraphics[height=0.85\textheight, keepaspectratio]
                      {ndvi_climate_spearman_ano_pt.jpg}                 
    \end{figure}   
    
  \end{columns}
  
\end{frame}

%%%%%%%%%% Slide 41 %%%%%%%%%%
\begin{frame}
  \frametitle{\normalsize{\textbf{
        (R2) Spearman's Rank Coefficient - Observed Data
  }}} 
  
  \begin{figure}[h!]
    \centering
    \includegraphics[width=0.49\textwidth, keepaspectratio]
                    {ndvi_climate_spearman_pt_ssr.jpg}
    \includegraphics[width=0.49\textwidth, keepaspectratio]
                    {ndvi_climate_spearman_pt_tp.jpg} \\
    \end{figure}   
  
\end{frame}

%%%%%%%%%% Slide 42 %%%%%%%%%%
\begin{frame}
  \frametitle{\normalsize{\textbf{
        (R2) Spearman's Rank Coefficient - Anomalies
  }}} 
  
  \begin{figure}[h!]
    \centering
    \includegraphics[width=0.49\textwidth, keepaspectratio]
                    {ndvi_climate_spearman_ano_pt_ssr.jpg}
    \includegraphics[width=0.49\textwidth, keepaspectratio]
                    {ndvi_climate_spearman_ano_pt_tp.jpg} \\
    \end{figure}   
  
\end{frame}

%%%%%%%%%% Slide 43 %%%%%%%%%%
\begin{frame}
  \frametitle{\normalsize{\textbf{
        (R2) Dimensionality Reduction and Clustering
  }}}

  \scriptsize{

    \textbullet \: \textbf{Highlights:} (1) Testing simple application of
    principal component analysis PCA followed by K-Means clustering; (2) These
    clusters can be a clue of the linear interactions between vegetation and
    climate; (3) A possible approach is to use an auto encoder neural network
    for non linear rediction; (4) Need for systematic selection of climate
    variables. 
    
    \begin{figure}[h!]
      \centering
      \includegraphics[width=0.495\textwidth, keepaspectratio]
                      {ndvi_climate_pca1_pt.jpg}
      \includegraphics[width=0.495\textwidth, keepaspectratio]
                      {ndvi_climate_kmeans_pt.jpg}                     
    \end{figure}
    
  }
  
\end{frame}

%%%%%%%%%% Slide 44 %%%%%%%%%%
\begin{frame}
  \frametitle{\normalsize{\textbf{
        (R2) Conclusions for Climate and Vegetation Analysis
  }}}

  \scriptsize{

    \textbullet \: It seems to be hard to model vegetation response to climate
    variables, specially if we use anomalies instead of observed data. \\[0.15cm]

    \textbullet \: Amazonia vegetation is anti-correlated with all the selected
    climate and surface variables. \\[0.15cm] 

    \textbullet \: North Amazonia vegetation does not have statistical
    significance in correlations with the predictors. \\[0.15cm]
    
    \textbullet \: We must build more features and or select another
    variables/datasets in order to seek for more predictive power. 
    
  }
    
\end{frame}

%%%%%%%%%%%%%%%%%%%%%%%%%%%%%%%%%%%%%%%%%%%%%%%%%%%%%%%%%%%%%%%%%%%%%%%%%%%%%%%
\subsection{(R3) Moisture Flux}
%%%%%%%%%%%%%%%%%%%%%%%%%%%%%%%%%%%%%%%%%%%%%%%%%%%%%%%%%%%%%%%%%%%%%%%%%%%%%%%

%%%%%%%%%% Slide 45 %%%%%%%%%%
\begin{frame}
  \frametitle{\normalsize{\textbf{
        (R3) Climatological Trajectories of Humid Air
  }}}
  
  \begin{figure}[h!]
    \centering
    \includegraphics[height=0.925\textheight, keepaspectratio]
                    {specific_humidity_and_moisture_flux_erainterim_mmeans_tra_pt.jpg} 
  \end{figure}    
  
\end{frame}

%%%%%%%%%% Slide 46%%%%%%%%%%
\begin{frame}
  \frametitle{\normalsize{\textbf{
        (R3) Water Variables along Climatological Trajectories (DJF)
  }}}
  
  \begin{figure}[h!]
    \centering
    \includegraphics[width=0.925\textwidth, keepaspectratio]
                    {trajectories_rainy.png} 
  \end{figure}    
  
\end{frame}

%%%%%%%%%% Slide 47 %%%%%%%%%%
\begin{frame}
  \frametitle{\normalsize{\textbf{
        (R3) Conclusions for Humid Trajectories
  }}}

  \scriptsize{

    \textbullet \: These trajectories show pronounced seasonality. \\[0.15cm]
    
    \textbullet \: Some of them fly over densely vegetated regions in Amazonia,
    while others go over less vegetation in Cerrado. \\[0.15cm]

    \textbullet \: This difference can be used to study the influence of
    vegetation in some atmospheric variables. \\[0.15cm]
   
  }
  
\end{frame}

%%%%%%%%%% Slide 48 %%%%%%%%%%
\begin{frame}
  \frametitle{\normalsize{\textbf{
        Presentation Structure
  }}}

  %%%%%%%%%%%%%%%%%%%%%%%%%%%%%%%%%%%%%%%%%%%%%%%%%%%%%%%%%%%%%%%%%%%%%%%%%%%%%
  \section{Final Considerations}
  %%%%%%%%%%%%%%%%%%%%%%%%%%%%%%%%%%%%%%%%%%%%%%%%%%%%%%%%%%%%%%%%%%%%%%%%%%%%%
  \tableofcontents[currentsection] 
  
\end{frame}

%%%%%%%%%%%%%%%%%%%%%%%%%%%%%%%%%%%%%%%%%%%%%%%%%%%%%%%%%%%%%%%%%%%%%%%%%%%%%%%
\subsection{Preliminary Conclusions}
%%%%%%%%%%%%%%%%%%%%%%%%%%%%%%%%%%%%%%%%%%%%%%%%%%%%%%%%%%%%%%%%%%%%%%%%%%%%%%%

%%%%%%%%%% Slide 49 %%%%%%%%%%
\begin{frame}
  \frametitle{\normalsize{\textbf{
        Preliminary Conclusions
  }}}

  \scriptsize{
    
    \textbullet \: (R1) Normalized Vegetation Difference Index:

    \begin{enumerate}
      
    \item South America is becoming greener, but not so much.
      
    \item We have found 9 vegetation clusters. We need to further evaluate if
      these clusters match land cover types.
      
    \end{enumerate}
      
    \textbullet \: (R2) Initial Investigations of Climate and Vegetation:

    \begin{enumerate}

    \item Amazonia vegetation is anti-correlated with all the selected climate
      and surface variables.
      
    \item North Amazonia vegetation does not have statistical significance in
      correlations with the predictors.

    \item It seems hard to model vegetation response to climate variables,
      specially if we use anomalies instead of observed data.
      
    \end{enumerate}
    
    \textbullet \: (R3) Moisture Flux:

    \begin{enumerate}

    \item These trajectories show pronounced seasonality.
    
    \item Some of them fly over densely vegetated regions in Amazonia,
    while others go over less vegetation in Cerrado. 

    \item This difference can be used to study the influence of
    vegetation in some atmospheric variables.
      
    \end{enumerate}
  }
  
\end{frame}

%%%%%%%%%%%%%%%%%%%%%%%%%%%%%%%%%%%%%%%%%%%%%%%%%%%%%%%%%%%%%%%%%%%%%%%%%%%%%%%
\subsection{Future Work}
%%%%%%%%%%%%%%%%%%%%%%%%%%%%%%%%%%%%%%%%%%%%%%%%%%%%%%%%%%%%%%%%%%%%%%%%%%%%%%%

%%%%%%%%%% Slide 50 %%%%%%%%%%
\begin{frame}
  \frametitle{\normalsize{\textbf{
        Future Work
  }}}

  \scriptsize{
    
    \textbullet \: (R1) Normalized Vegetation Difference Index:

    \begin{enumerate}

    \item We are reviewing literature and searching for physical
      interpretations.

    \item These results have potential for publication.
      
    \end{enumerate}
      
    \textbullet \: (R2) Initial Investigations of Climate and Vegetation:

    \begin{enumerate}

    \item We are working on predictive models (RFs, LSTMs) and the related
      methodology in order to estimate  causality of climate variables on
      vegetation. 
      
    \item Work on dimensionality reduction + clusterization in order to find
      vegetation-climate regimes. 
      
    \item These results have potential for publication.      
      
    \end{enumerate}
    
    \textbullet \: (R3) Moisture Flux:

    \begin{enumerate}

    \item The trajectories can be thought as time series.
      
    \item Possibility to study the reverse causal influence of vegetation on
      some atmospheric variables.

    \item These possible results has potential for publication.
      
    \end{enumerate}
  }
  
\end{frame}

%%%%%%%%%% Slide 51 %%%%%%%%%%
\begin{frame}
  \frametitle{\normalsize{\textbf{The End}}}

  \section{}
  \begin{center}
    \huge{Thank You !}
  \end{center}

\end{frame}

% Hidden slides from now on.
\appendix

%%%%%%%%%% Slide XX %%%%%%%%%%
\begin{frame}
  \frametitle{\normalsize{\textbf{
        Presentation Structure
  }}}

  %%%%%%%%%%%%%%%%%%%%%%%%%%%%%%%%%%%%%%%%%%%%%%%%%%%%%%%%%%%%%%%%%%%%%%%%%%%%%
  \section{Hidden Material}
  %%%%%%%%%%%%%%%%%%%%%%%%%%%%%%%%%%%%%%%%%%%%%%%%%%%%%%%%%%%%%%%%%%%%%%%%%%%%%
  \tableofcontents[currentsection] 
  
\end{frame}

\subsection{}
\begin{thebibliography}{99}
  
  %%%%%%%%%% Slide XX %%%%%%%%%%%
  \begin{frame}
    \frametitle{\normalsize{\textbf{
          Bibliography 
    }}}

    \scriptsize{
      
    \bibitem{}
      P. A. Dirmeyer, P. Gentine, M. B. Ek, G. Balsamo, “Land surface processes
      relevant to sub-seasonal to seasonal (S2S) prediction” in Sub-Seasonal to
      Seasonal Prediction, A. W. Robertson, F. Vitart, Eds. (Elsevier, 2019),
      pp. 165–181.

    \bibitem{}
      Overpeck, J. T., Meehl, G. A., Bony, S., and Easterling,
      D. R. (2011). Climate data challenges in the 21st century. science,
      331(6018):700–702.
      
    \bibitem{}
      Reichstein, M., Camps-Valls, G., Stevens, B., Jung, M., Denzler, J.,
      Carvalhais, N., et al. (2019). Deep learning and process understanding for
      data-driven earth system science. Nature, 566(7743):195.
      
    \bibitem{}
      Runge, J., Bathiany, S., Bollt, E., Camps-Valls, G., Coumou, D., Deyle,
      E., Glymour, C., Kretschmer, M., Mahecha, M. D., Muñoz-Marí, J., et
      al. (2019a). Inferring causation from time series in earth system
      sciences. Nature communications, 10(1):2553. 

    \bibitem{}
      Loveland, T., J. Brown, D. Ohlen, B. Reed, Z. Zhu, L. Yang, and S. Howard
      . 2009.ISLSCP II IGBP DISCover and SiB Land Cover, 1992-1993. In Hall,
      Forrest G., G. Collatz, B. Meeson, S. Los, E. Brown de Colstoun, and
      D. Landis (eds.). ISLSCP Initiative II Collection. Data set. Available
      on-line [http://daac.ornl.gov/] from Oak Ridge National Laboratory
      Distributed Active Archive Center, Oak Ridge, Tennessee,
      U.S.A. doi:10.3334/ORNLDAAC/930 

    } 

  \end{frame}

  %%%%%%%%%% Slide XX %%%%%%%%%%%  
  \begin{frame}
    \frametitle{\normalsize{\textbf{
          Bibliography 
    }}}

    \scriptsize{
      
    \bibitem{}
      Pinzon, J. and Tucker, C. (2014). A non-stationary 1981–2012 avhrr ndvi3g
      time series. Remote Sensing, 6(8):6929–6960. 

    \bibitem{}
      Dee, D. P., Uppala, S. M., Simmons, A., Berrisford, P., Poli, P.,
      Kobayashi, S., Andrae, U., Balmaseda, M., Balsamo, G., Bauer, d. P., et
      al. (2011). The era-interim rea- nalysis: Configuration and performance of
      the data assimilation system. Quarterly Journal of the royal
      meteorological society, 137(656):553–597.
      
    \bibitem{}
      Chollet, F. et al. (2018). Keras: The python deep learning
      library. Astrophysics Source Code Library.
      
    \bibitem{}
      Mehta, P., Bukov, M., Wang, C.-H., Day, A. G., Richardson, C., Fisher,
      C. K., and Schwab, D. J. (2019). A high-bias, low-variance introduction to
      machine learning for physicists. Physics reports.

    \bibitem{}
      Géron, A. (2019). Hands-On Machine Learning with Scikit-Learn, Keras, and
      TensorFlow: Concepts, Tools, and Techniques to Build Intelligent
      Systems. O’Reilly Media.
      
    } 

  \end{frame}

  %%%%%%%%%% Slide XX %%%%%%%%%%%  
  \begin{frame}
    \frametitle{\normalsize{\textbf{
          Bibliography 
    }}}

    \scriptsize{
      
    \bibitem{}
      Olah, C. (2015). Understanding lstm networks. Available at
      \url{https://colah.github.io/posts/2015-08-Understanding-LSTMs/}. Accessed
      15 Jan 2020.
      
    \bibitem{}
      Decubber, S. (2017). Spatiotemporal optimization of granger causality
      methods for climate change attribution. Master’s thesis, Department of
      Applied Mathematics, Computer Science and Statistics of Faculty of
      Sciences, Ghent University. 
      
    \bibitem{}
      Papagiannopoulou, C., Gonzalez Miralles, D., Decubber, S., Demuzere, M.,
      Verhoest, N., Dorigo, W. A., and Waegeman, W. (2017a). A non-linear
      granger-causality framework to investigate climate-vegetation
      dynamics. Geoscientific Model Development, 10(5):1945–1960. 
 
    \bibitem{}
      Papagiannopoulou, C., Miralles, D. G., Dorigo, W. A., Verhoest, N. E. C.,
      Depoorter, M., and Waegeman, W.: Vegetation anomalies caused by antecedent
      precipitation in most of the world, Environ. Res. Lett.,
      doi:10.1088/1748-9326/aa7145, 2017b. 

    \bibitem{}
      Huffman, G. J., Bolvin, D. T., Nelkin, E. J., Wolff, D. B., Adler, R. F.,
      Gu, G., Hong, Y., Bowman, K. P., and Stocker, E. F. (2007). The trmm
      multisatellite precipitation analysis (tmpa): Quasi-global, multiyear,
      combined-sensor precipitation estimates at fine scales. Journal of
      hydrometeorology, 8(1):38–55.
      
    } 

  \end{frame}

  %%%%%%%%%% Slide XX %%%%%%%%%%%  
  \begin{frame}
    \frametitle{\normalsize{\textbf{
          Bibliography 
    }}}

    \scriptsize{
      
    \bibitem{}
      Knüsel, B., Zumwald, M., Baumberger, C., Hadorn, G. H., Fischer, E. M.,
      Bresch, D. N., and Knutti, R. (2019). Applying big data beyond small
      problems in climate research. Nature Climate Change, 9(3):196.

    \bibitem{}
      Nobre, C (2004). Interações entre clima e vegetação na Amazônia: do último
      período glacial até o clima do futuro. III Conferência científica do LBA.
      
    }
    
  \end{frame}
  
\end{thebibliography}

%%%%%%%%%% Slide XX %%%%%%%%%%
\begin{frame}
  \frametitle{\normalsize{\textbf{
        Data Sources in Detail
  }}} 
    
  \begin{figure}[h!]
    \centering
    \includegraphics[height=0.925\textheight, keepaspectratio]{erainterim2.png} 
  \end{figure}
  
\end{frame}

%%%%%%%%%% Slide XX %%%%%%%%%%
\begin{frame}
  \frametitle{\normalsize{\textbf{
       Supervised Learning
  }}} 

  \scriptsize{

    \begin{columns}
      \column{0.55\linewidth}
      
      \textbf{1)} Discovery of mapping from input observations to known output
      observations: 

      $$y = f(X) + \varepsilon $$

      \textbf{2)} Data coming from this function:

      $$\mathcal{D} = (X_i, y_i); \: \: i=1, 2, 3, ..., N$$

      \textbf{3)} Candidate function $g(\theta)$ for best approximating $f(x)$,
      where parameters $\theta$ must be found through \emph{learning}.

      $$
      g(\theta) \: \text{is my model}
      $$

      \textbf{4)} Split data into \textcolor{red}{train},
      \textcolor{green}{validation}, and \textcolor{blue}{test}. \\[0.075cm]

      \textbf{5)} Choose a cost function $C(\mathcal{D}, g(\theta))$ that
      allows us to judge how well the model explains the observations. \\

      \column{0.55\linewidth}
      
      \textbf{6)} The model is fit by finding the values of $\theta$ that
      minimizes the cost function on \textcolor{red}{train} data. \\[0.01cm]

      \textbf{7)} Search for \textbf{hyperparameters} using
      \textcolor{green}{validation} data. \\[0.01cm]

      \textbf{8)} Final evaluation of \textbf{model} using
      \textcolor{blue}{test} data.
      
      \begin{figure}[h!]
        \centering
        \includegraphics[width=0.95\textwidth, keepaspectratio]{schematic.png} 
        \caption*{\scriptsize{
            \citat{Mehta}{2019}
        }}
    \end{figure}
    
    \end{columns}
    
  }
    
\end{frame}

%%%%%%%%%% Slide XX %%%%%%%%%%
\begin{frame}
  \frametitle{\normalsize{\textbf{
       Vegetation Clustering by K-Means
  }}}
  
    \begin{columns}
      \column{0.55\linewidth}

      \begin{figure}[h!]
        \centering
        \includegraphics[width=0.95\textwidth, keepaspectratio]
                        {ndvi_gimms_clustering_scaled_map.jpg}                      
      \end{figure}
    
      \column{0.55\linewidth}
      
      \begin{figure}[h!]
        \centering
        \includegraphics[width=0.95\textwidth, keepaspectratio]
                        {ndvi_gimms_clustering_scaled_pt.jpg}          
        \includegraphics[width=0.95\textwidth, keepaspectratio]
                        {ndvi_gimms_clustering_scaled_time_series.jpg}
      \includegraphics[width=0.95\textwidth, keepaspectratio]
                    {ndvi_gimms_clustering_scaled_boxplot.jpg}    
    \end{figure}   
    
  \end{columns}
 
\end{frame}

%%%%%%%%%% Slide XX %%%%%%%%%%
\begin{frame}
  \frametitle{\normalsize{\textbf{
        Predictive Modelling with Random Forests
  }}}

  \scriptsize{

    \textbullet \: \textbf{Highlights:} (1) Select a region in South Amazonia
    (7200 NDVI time series); (2) Use mean of that; (3) Fitting 5 folds for each
    of 180 candidates, totalling 900 fits; (4) 9 min on 4 cpus; (5) Overfitting.
    
    \begin{figure}[h!]
      \centering
      \includegraphics[width=0.70\textwidth, keepaspectratio]
                      {rf_ndvi.png}
      \includegraphics[width=0.29\textwidth, keepaspectratio]
                      {kfolds2.png}                        
      \includegraphics[width=0.35\textwidth, keepaspectratio]
                      {rf_ndvi2.png}  
      \includegraphics[width=0.35\textwidth, keepaspectratio]
                      {rf_ndvi3.png}                                     
    \end{figure}   
  }
    
\end{frame}

%%%%%%%%%%%%%%%%%%%%%%%%%%%%%%%%%%%%%%%%%%%%%%%%%%%%%%%%%%%%%%%%%%%%%%%%%%%%%%%%
\end{document}
