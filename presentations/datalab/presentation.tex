\documentclass[11pt]{beamer}

\mode<presentation> {
  \usetheme{Singapore} 
  \usecolortheme{seagull}
  \setbeamertemplate{section in toc}[ball unnumbered]
  \setbeamerfont{section in toc}{size=\scriptsize}
  \setbeamertemplate{subsection in toc}[ball unnumbered]
  \setbeamerfont{subsection in toc}{size=\scriptsize}
  \usebeamercolor[fg]{page number in head/foot}
  \setbeamertemplate{footline}[page number]
  \setbeamerfont{page number in head/foot}{size=\large}
  \setbeamercolor{page number in head/foot}{fg=black}
  \setbeamercolor{block title}{bg=blue!50, fg=black}
  \setbeamertemplate{navigation symbols}{}
}

\usepackage{graphicx}
\usepackage{booktabs}
\usepackage{newverbs}
\usepackage[utf8]{inputenc}
\usepackage[brazil]{babel}
\usepackage{amsmath}
\usepackage[style=authoryear]{biblatex}
\usepackage{xcolor}
\usepackage{anyfontsize}
\usepackage{amsfonts}
\usepackage{listings}
\usepackage{amssymb}
\usepackage{url}
\usepackage[version=3]{mhchem}
\usepackage{caption}
\usepackage{multirow}
\usepackage{amsthm}
\usepackage{pythonhighlight}
\usepackage{hyperref}
\usepackage{tikz}
\usepackage{bm}
\usepackage[export]{adjustbox}
\usetikzlibrary{arrows, automata}
\graphicspath{
  {/home/alex/Dropbox/doutorado/figuras/}
  {/home/alex/Dropbox/repositories/phd/images/}
}

\hypersetup{
  colorlinks=true,
  linkcolor=black,
  filecolor=cyan,      
  urlcolor=magenta,
}

\newcommand{\btVFill}{\vskip0pt plus 1filll}

\usepackage{ragged2e}
\justifying
\renewcommand{\raggedright}{\leftskip=0pt \rightskip=0pt plus 0cm}
\addtobeamertemplate{block begin}{}{\justifying\setlength{\parindent}{2em}}

\newverbcommand{\bverb}{\color{blue}}{}
\newverbcommand{\gverb}{\color{green}}{}
\newverbcommand{\kverb}{\color{black}}{}
\newverbcommand{\rverb}{\color{red}}{}

\makeatletter
\newlength\beamerleftmargin
\setlength\beamerleftmargin{\Gm@lmargin}
\makeatother

% João (1988)
\newcommand{\citatu}[2]{\textbf{\textcolor{blue}{#1 (#2)}}}        

% (João, 1988)
\newcommand{\citapu}[2]{\textbf{\textcolor{blue}{(#1, #2)}}}       

 % João et al. (1988)
\newcommand{\citat}[2]{\textbf{\textcolor{blue}{#1 et al. (#2)}}}

% (João et al., 1988)
\newcommand{\citap}[2]{\textbf{\textcolor{blue}{(#1 et al., #2)}}} 

 % João et al., 1988 [Para citar varios]
\newcommand{\citav}[2]{\textbf{\textcolor{blue}{#1 et al., #2}}}

 % João, 1988 [Para citar varios]
\newcommand{\citavu}[2]{\textbf{\textcolor{blue}{#1, #2}}}

%%%%%%%%%%%%%%%%%%%%%%%%%%%%%%%%%%%%%%%%%%%%%%%%%%%%%%%%%%%%%%%%%%%%%%%%%%%%%%%%
\begin{document}

\title[
  \textbf{\textcolor{blue}{
    Presentation
  }} 
]{\\[0.01cm]
  \large{\textbf{\textcolor{black}{
    Datalab Journal Club 
  }}} \\
  
  \small{\textcolor{black}{
    \emph{Presentation}
  }} \\[0.50cm]

  \large{\textbf{\textcolor{black}{
	  Long Term Trends for Vegetation in South America 
  }}} \\
    \small{PhD studies}
} 

\author{
  \textcolor{black}{\textbf{
      Alex Araujo 
  }} \\
    \small{Alexandre Correia, Pedro Dias, Henrique Barbosa}
} 

\institute[
  University of São Paulo
]{
  \textcolor{black}{
    University of São Paulo - USP \\ 
    \medskip
    \url{alex.araujo@usp.br}
  }
}

\date{
  \textcolor{black}{
    \today
  }
}

% Template for slide.
%%%%%%%%%% Slide number %%%%%%%%%%
% \begin{frame}[fragile]
%   \frametitle{\normalsize{\textbf{
%     Title
%   }}} 
%
%   \scriptsize{  
%
%   }
% \end{frame}

%%%%%%%%%% Slide 1 %%%%%%%%%%
{
  \usebackgroundtemplate{
    \centering
    \includegraphics[
      width=\paperwidth, height=0.92\paperheight
    ]{background.jpg}
  }
  \begin{frame}
    \titlepage 
  \end{frame}
}

%%%%%%%%%% Slide 2 %%%%%%%%%%
\begin{frame}
  \frametitle{\normalsize{\textbf{
    Presentation Structure
  }}}
  
  \tableofcontents
  
\end{frame}

%%%%%%%%%%%%%%%%%%%%%%%%%%%%%%%%%%%%%%%%%%%%%%%%%%%%%%%%%%%%%%%%%%%%%%%%%%%%%%%
\section{Introduction}
%%%%%%%%%%%%%%%%%%%%%%%%%%%%%%%%%%%%%%%%%%%%%%%%%%%%%%%%%%%%%%%%%%%%%%%%%%%%%%%

%%%%%%%%%% Slide 3 %%%%%%%%%%
\begin{frame}[fragile]
  \frametitle{\normalsize{\textbf{
    Abstract
  }}} 

  \scriptsize{  

    There is a growing body of evidence showing that \textbf{\textcolor{green}{
    global and regional vegetation is greening at least since the beginning of 
    80’s, and it is considered a signature of anthropogenic climate change}}. 
    Even though this is a very important scientific question, 
    \textbf{\textcolor{red}{we note a gap in specific literature for what is 
    happening in South American continent regarding vegetation trends}}. 
    \textbf{\textcolor{blue}{The main goal of this work is to better understand 
    the spatial patterns and their drivers for long term trends in vegetation 
    dynamics of South America}}. \textbf{\textcolor{cyan}{For doing that, 
    we analyze a widely used Normalized Difference Vegetation Index (NDVI) 
    data set by means of a recently developed methodology for statistical 
    significance that can simultaneously assess common problems found in 
    environmental data analysis: (1) the appropriate statistics to measure 
    monotonic long term trends, (2) temporal and (3) spatial autocorrelations, 
    and (4) multiple hypothesis tests}}. Our results show a strong signal in 
    Brazilian areas notoriously affected by human land use change, places where 
    the vegetation show browning trends (Rondônia, Amazonas River, Rio de Janeiro
     - São Paulo axis, and Arc of Deforestation), as well as great contiguous 
    arid lands in Brazilian Northeast region and northern Argentina, where 
    desertification is taking place. On the other hand, we observed areas with 
    greening trends in Brazilian States of São Paulo and Minas Gerais, 
    together with a long belt in Pacific coastal region, beginning in Peru
    and going through Equator, Colombia, and Venezuela. \textbf{\textcolor{black}{
    Taking into account only the results with statistical significance, we show 
    that South America as a whole is becoming greener over the last three 
    decades, in agreement with what has been observed across the globe.}}

  }
\end{frame}

%%%%%%%%% Slide 4 %%%%%%%%%%
\begin{frame}[fragile]
  \frametitle{\normalsize{\textbf{
    How Does NDVI Work?
  }}} 

  \scriptsize{  

    Plants absorb radiation at red, but they reflect at near
    infrared in order to avoid excessive heating of its internal tissues, 
    where $\rho_{\text{infra}}$ are $\rho_{\text{red}}$ are reflectances:

    \begin{equation}
      \boxed{\text{NDVI} = \frac{\rho_{\text{infra}} -
        \rho_{\text{red}}}{\rho_{\text{infra}} + \rho_{\text{red}}} \nonumber}
    \end{equation} 

    \begin{figure}[h!]
      \centering
      \includegraphics[width=0.46\linewidth, keepaspectratio]{ndvi1.jpg}
      \includegraphics[width=0.46\linewidth, keepaspectratio]{ndvi2.jpg} \\
      \includegraphics[width=0.46\linewidth, keepaspectratio]{ndvi3.jpg}
      \includegraphics[width=0.46\linewidth, keepaspectratio]{ndvi4.jpg}
      \caption*{\scriptsize{
          Source: \href{https://eos.com/blog/ndvi-faq-all-you-need-to-know-about-ndvi/}{NDVI FAQ: All You Need To Know About Index}.
      }}
    \end{figure}

  }
\end{frame}

%%%%%%%%%% Slide 5 %%%%%%%%%%
\begin{frame}
  \frametitle{\normalsize{\textbf{
        Weather and Climate Data
  }}}

  \scriptsize{
    
    \begin{figure}[h!]
      \centering
      \includegraphics[height=0.25\textheight, keepaspectratio]{table_ml.png} \\
      \includegraphics[width=0.75\textwidth, keepaspectratio]{hypercubes.png} 
      \includegraphics[width=0.20\textwidth, keepaspectratio]{grid.png}
    \end{figure}
    
    \textbullet \: \textbf{\textcolor{blue}{NetCDF}} (Network Common Data Form) is a set of software 
    libraries and machine-independent data formats that support the creation, 
    access, and sharing of array-oriented scientific data. It is also a community 
    standard for sharing scientific data. 

    \textbullet \: Interfaces available for C, C++, Java, Fortran, 
    \textbf{\textcolor{blue}{Python}}, IDL, MATLAB, R, Ruby, and Perl.

    \textbullet \: Data in netCDF format is \textbf{\textcolor{blue}{self-describing}}, 
    \textbf{\textcolor{blue}{portable}}, \textbf{\textcolor{blue}{scalable}}, 
    \textbf{\textcolor{blue}{appendable}}, \textbf{\textcolor{blue}{sharable}}, and 
    \textbf{\textcolor{blue}{archivable}}. 
  }
\end{frame}

%%%%%%%%%% Slide 6 %%%%%%%%%%
\begin{frame}
  \frametitle{\normalsize{\textbf{
        Tools for Climate Science
  }}} 

  \scriptsize{
    
    \begin{figure}[h!]
      \centering
      \includegraphics[width=\textwidth, height=\textheight,keepaspectratio]
                      {tools.jpg}
    \end{figure}
    
  }
\end{frame}

%%%%%%%%%%%%%%%%%%%%%%%%%%%%%%%%%%%%%%%%%%%%%%%%%%%%%%%%%%%%%%%%%%%%%%%%%%%%%%%
\section{Methodology}
%%%%%%%%%%%%%%%%%%%%%%%%%%%%%%%%%%%%%%%%%%%%%%%%%%%%%%%%%%%%%%%%%%%%%%%%%%%%%%%

%%%%%%%%%% Slide 7 %%%%%%%%%%
\begin{frame}
  \frametitle{\normalsize{\textbf{
        Data Source for Vegetation (GIMMS NDVI3g 1981 - 2015)
  }}} 

  \scriptsize{
    
    \begin{columns}
      \column{0.50\linewidth}

      \textbullet \: Vegetation indices are radiometric
      measures of photosynthetically active radiation absorbed by chlorophyll in
      plants. \\

      \textbullet \:They are surrogates of vegetation development as well as
      land use change. \\

      \textbullet \: \textbf{N}ormalized \textbf{D}ifference \textbf{V}egetation
      \textbf{I}ndex NDVI. \\[0.10cm]  
      
      \textbullet \: Plants absorb radiation at red, but they reflect at near
      infrared in order to avoid excessive heating of its internal tissues:

      \begin{equation}
        \boxed{\text{NDVI} = \frac{\rho_{\text{infra}} -
          \rho_{\text{red}}}{\rho_{\text{infra}} + \rho_{\text{red}}} \nonumber}
      \end{equation} 
      where $\rho_{\text{infra}}$ are $\rho_{\text{red}}$ are reflectances. 
      
      \column{0.60\linewidth}

      \begin{figure}[h!]
        \centering
        \includegraphics[width=\linewidth, keepaspectratio]{ndvi_gimms.png}
        \caption*{\scriptsize{
            \citat{Pinzon and Tucker}{2014}
        }}
      \end{figure}
      
    \end{columns}

    \vspace{0.10cm}
    
    \textbf{\textbullet \: Metadata:} (1) netCDF format; (2) Global domain;
    \textcolor{red}{(3) 1/12 degree lat-2160 lon-4320}; (4) Daily time step;
    \textcolor{red}{(5) Biweekly from 1981/07 to 2015/12}; (6) AVHRR input data. 
   
  }
\end{frame}

%%%%%%%%% Slide 8 %%%%%%%%%%
\begin{frame}[fragile]
  \frametitle{\normalsize{\textbf{
    Trend Detection in Climate Data
  }}} 

  \scriptsize{  

    \textbullet \: When analyzing \textbf{\textcolor{red}{statistical 
    significance of trends in gridded environmental data over a large region}}, 
    we face some challenges, and the four most important ones are: 
    
    \begin{enumerate}
    \item \textbf{Use of an adequate estimator for monotonic trends};
    \item Temporal autocorrelation;
    \item Spatial autocorrelation;
    \item The problem of multiple hypothesis test.  
    \end{enumerate}

    \textbullet \: \textbf{\textcolor{blue}{Theil-Sen estimator}}, 
    a \textbf{\textcolor{green}{non-parametric}} robust analysis method to 
    obtain monotonic trends $\beta$:

    \begin{equation*}
      \boxed{
        \beta = \text{median} \left( \frac{\text{NDVI}_j - \text{NDVI}_i}{j - i}
        \right); \: \: \: \forall j > i
      }
    \end{equation*}

    where $i$ and $j$ are time series indexes, $\text{NDVI}_i$ and 
    $\text{NDVI}_j$ denote the NDVI values at time $i$ and $j$, respectively. 
    The estimator $\beta$ is the median of slopes in all the $n (n - 1) / 2$ 
    possible pair combinations of data points for that spatial grid point. 
    If $\beta > 0$, it indicates that the time series has an increasing 
    trend, and vice-versa.

  }
\end{frame}

%%%%%%%%% Slide 9 %%%%%%%%%%
\begin{frame}[fragile]
  \frametitle{\normalsize{\textbf{
    Trend Detection in Climate Data
  }}} 

  \scriptsize{  

    \textbf{\textcolor{red}{
      1. Use of an adequate estimator for monotonic trends;
    }}

    \begin{figure}[h!]
      \centering
      \includegraphics[width=0.90\linewidth, keepaspectratio]{datalab_ts.jpg} \\
      \includegraphics[width=0.90\linewidth, keepaspectratio]{datalab_ts_dist.jpg}
    \end{figure}

  }
\end{frame}

%%%%%%%%% Slide 10 %%%%%%%%%%
\begin{frame}[fragile]
  \frametitle{\normalsize{\textbf{
    Trend Detection in Climate Data
  }}} 

  \scriptsize{  

    \textbf{\textcolor{red}{
      2. Temporal autocorrelation;
    }}

    \begin{equation*}
      \boxed{
        r_k = 
        \displaystyle\sum_{t=k+1}^{n-k} (x_{t-k} - \bar{x})(x_t - \bar{x})) 
        \Bigg/
        \displaystyle\sum_{t=1}^{n} (x_t - \bar{x})^2 = 
        \text{\textbf{Corr}} (x_t, x_{t-k}), \: k=1,2, \cdots
      }
    \end{equation*}    

    \begin{figure}[h!]
      \centering
      \includegraphics[width=0.69\linewidth, keepaspectratio]{datalab_ts.jpg} \\
      \includegraphics[width=0.69\linewidth, keepaspectratio]{datalab_temporal.jpg}
    \end{figure}

  }
\end{frame}

%%%%%%%%% Slide 11 %%%%%%%%%%
\begin{frame}[fragile]
  \frametitle{\normalsize{\textbf{
    Trend Detection in Climate Data
  }}} 

  \scriptsize{  

    \textbf{\textcolor{red}{
      3. Spatial autocorrelation;
    }}

    \begin{figure}[h!]
      \centering
      \includegraphics[height=0.9\textheight, keepaspectratio]{datalab_spatial.jpg} \\
    \end{figure}

  }
\end{frame}

%%%%%%%%%% Slide 12 %%%%%%%%%%
\begin{frame}
  \frametitle{\normalsize{\textbf{
    Trend Detection in Climate Data  
  }}} 

  \scriptsize{  

    \textbf{\textcolor{red}{
      4. The problem of multiple hypothesis test.
    }} \\[0.10cm]

    \textbullet \: Assume we wil perform $n$ \textbf{independent and
     simultaneous hypothesis tests} each one at a desired $\alpha$ level. \\[0.10cm] 

    \textbullet \: The probability of obtaining \textbf{at least one} 
    significant result by chance (\textbf{false positive}) greatly increases. \\[0.10cm]

    \textbullet \: This probability of at least one false positive
    among a \emph{family} of tests is called the 
    \textbf{Familywise error rate (FWER)} 

    \begin{equation*}
      P_0 + \underbrace{P_1 + P_2 + P_3 + P_4 + \cdots + P_n}_{\text{FWER}} = 1 
      \Rightarrow \text{FWER} = 1 - P_0  
      \Rightarrow \boxed{FWER = 1 - (1 - \alpha)^n}
    \end{equation*}

    \begin{figure}[h!]
      \centering
      \includegraphics[height=0.35\textheight, keepaspectratio]{datalab_sa.jpg}
      \includegraphics[width=0.70\textwidth, keepaspectratio]{datalab_fwer.jpg}
    \end{figure}

  }
\end{frame}

%%%%%%%%% Slide 13 %%%%%%%%%%
\begin{frame}[fragile]
  \frametitle{\normalsize{\textbf{
    Data Permutation with Criteria
  }}} 

  \tiny{  

    \begin{table}[h!]
      \centering
      \caption*{\tiny \textbullet \: One example of the shuffling procedure intended 
      to generate \textbf{surrogate time series} that will be used to build 
      \textbf{null distribution} in trend significance tests. Each grid point 
      is a time series with \textbf{828} data points with time indexes 
      \textbf{from 1981-07-01 to 2015-12-15}. After the permutation of
      the natural order, we obtain one shuffled \textbf{\textcolor{red}{
        time series that preserves much of the original temporal 
        autocorrelation.}}}
      \begin{tabular}{|c||c|c|}
        \hline
        \textbf{Count} & \textbf{Original} & \textbf{Shuffled} \\ \hline \hline
        1 & 1981-07-01 & 1986-07-01 \\ \hline
        2 & 1981-07-15 & 1996-07-15 \\ \hline
        3 & 1981-08-01 & 2003-08-01 \\ \hline
        4 & 1981-08-15 & 2000-08-15 \\ \hline
        5 & 1981-09-01 & 2003-09-01 \\ \hline
        6 & 1981-09-15 & 2013-09-15 \\ \hline
        7 & 1981-10-01 & 2006-10-01 \\ \hline
        8 & 1981-10-15 & 2007-10-15 \\ \hline
        9 & 1981-11-01 & 1986-11-01 \\ \hline
        10 & 1981-11-15 & 1989-11-15 \\ \hline
        11 & 1981-12-01 & 2002-12-01 \\ \hline
        12 & 1981-12-15 & 1993-12-15 \\ \hline
        13 & 1982-01-01 & 1986-01-01 \\ \hline
        14 & 1982-01-15 & 1993-01-15 \\ \hline
        \vdots & \vdots & \vdots \\ \hline
        825 & 2015-11-01 & 1984-11-01 \\ \hline
        826 & 2015-11-15 & 1984-11-15 \\ \hline
        827 & 2015-12-01 & 2001-12-01 \\ \hline
        828 & 2015-12-15 & 2008-12-15 \\ \hline
      \end{tabular}
      \label{tab:shuffle}
    \end{table}

  }
\end{frame}

%%%%%%%%% Slide 14 %%%%%%%%%%
\begin{frame}[fragile]
  \frametitle{\normalsize{\textbf{
    Data Permutation with Criteria
  }}} 

  \scriptsize{  

    \begin{figure}[h!]
      \centering
      \includegraphics[height=0.925\textheight, keepaspectratio]{datalab_permutation1.jpg}
    \end{figure}

  }
\end{frame}

%%%%%%%%% Slide 15 %%%%%%%%%%
\begin{frame}[fragile]
  \frametitle{\normalsize{\textbf{
    Data Permutation with Criteria
  }}} 

  \scriptsize{  

    \begin{figure}[h!]
      \centering
      \includegraphics[height=0.925\textheight, keepaspectratio]{datalab_permutation2.jpg}
    \end{figure}

  }
\end{frame}

%%%%%%%%% Slide 16 %%%%%%%%%%
\begin{frame}[fragile]
  \frametitle{\normalsize{\textbf{
    Data Permutation with Criteria
  }}} 

  \scriptsize{  

    \begin{figure}[h!]
      \includegraphics[width=\textwidth, keepaspectratio]{datalab_permutation3.jpg}
    \end{figure}

  }
\end{frame}

%%%%%%%%% Slide 17 %%%%%%%%%%
\begin{frame}[fragile]
  \frametitle{\normalsize{\textbf{
    Data Permutation with Criteria
  }}} 

  \scriptsize{  

    \begin{figure}[h!]
      \centering
      \includegraphics[width=0.925\textwidth, keepaspectratio]{datalab_permutation4.jpg} \\
      \includegraphics[width=1.05\textwidth, keepaspectratio]{datalab_permutation5.jpg}
      \caption*{\citav{Cortés}{2020}}
    \end{figure}

  }
\end{frame}

%%%%%%%%%%%%%%%%%%%%%%%%%%%%%%%%%%%%%%%%%%%%%%%%%%%%%%%%%%%%%%%%%%%%%%%%%%%%%%%
\section{Results}
%%%%%%%%%%%%%%%%%%%%%%%%%%%%%%%%%%%%%%%%%%%%%%%%%%%%%%%%%%%%%%%%%%%%%%%%%%%%%%%

%%%%%%%%% Slide 18 %%%%%%%%%%
\begin{frame}[fragile]
  \frametitle{\normalsize{\textbf{
    Summary Statistics
  }}} 

  \scriptsize{  

    \begin{figure}[h!]
      \centering
      \includegraphics[height=0.93\textheight, keepaspectratio]{results1.jpg} \\
    \end{figure}

  }
\end{frame}

%%%%%%%%% Slide 19 %%%%%%%%%%
\begin{frame}[fragile]
  \frametitle{\normalsize{\textbf{
    Summary Statistics
  }}} 

  \scriptsize{  

    \begin{figure}[h!]
      \centering
      \includegraphics[height=0.93\textheight, keepaspectratio]{monthly_means.jpg} \\
    \end{figure}

  }
\end{frame}

%%%%%%%%% Slide 20 %%%%%%%%%%
\begin{frame}[fragile]
  \frametitle{\normalsize{\textbf{
    Forest, Savanna, and Treeless 
  }}} 

  \scriptsize{  

    \begin{figure}[h!]
      \centering
      \includegraphics[width=5.5cm]{results2.jpg}
      \includegraphics[width=3.5cm]{results3.jpg} \\
      \includegraphics[width=9.925cm]{results4.jpg}
    \end{figure}
  
  }
\end{frame}

%%%%%%%%% Slide 21 %%%%%%%%%%
\begin{frame}[fragile]
  \frametitle{\normalsize{\textbf{
    Time Peaking
  }}} 

  \scriptsize{  

    \begin{figure}[h!]
      \centering
      \includegraphics[height=0.76\textheight, keepaspectratio]{results5.jpg}
      \includegraphics[height=0.76\textheight, keepaspectratio]{results6.jpg}
    \end{figure}

  }
\end{frame}

%%%%%%%%% Slide 22 %%%%%%%%%%
\begin{frame}[fragile]
  \frametitle{\normalsize{\textbf{
    Long Term Trends of South American Vegetation Activity
  }}} 

  \scriptsize{  

    \begin{figure}[h!]
      \centering
      \includegraphics[width=0.925\linewidth, keepaspectratio]{results7.jpg}
    \end{figure}

  }
\end{frame}

%%%%%%%%% Slide 23 %%%%%%%%%%
\begin{frame}[fragile]
  \frametitle{\normalsize{\textbf{
    Long Term Trends of South American Vegetation Activity
  }}} 

  \scriptsize{  

    \begin{figure}[h!]
      \centering
      \includegraphics[width=\linewidth, keepaspectratio]{results8.jpg}
    \end{figure}

  }
\end{frame}

%%%%%%%%%%%%%%%%%%%%%%%%%%%%%%%%%%%%%%%%%%%%%%%%%%%%%%%%%%%%%%%%%%%%%%%%%%%%%%%
\section{Work in Progress}
%%%%%%%%%%%%%%%%%%%%%%%%%%%%%%%%%%%%%%%%%%%%%%%%%%%%%%%%%%%%%%%%%%%%%%%%%%%%%%%

%%%%%%%%% Slide 24 %%%%%%%%%%
\begin{frame}[fragile]
  \frametitle{\normalsize{\textbf{
    Amazonia - Different Instruments from Different Satellites
  }}} 

  \scriptsize{  

    \begin{figure}[h!]
      \centering
      \includegraphics[width=\linewidth, keepaspectratio]{wip1.jpg}
    \end{figure}

  }
\end{frame}

%%%%%%%%% Slide 25 %%%%%%%%%%
\begin{frame}[fragile]
  \frametitle{\normalsize{\textbf{
    Amazonia - Different Instruments from Different Satellites
  }}} 

  \scriptsize{  

    \begin{figure}[h!]
      \centering
      \includegraphics[width=0.925\linewidth, keepaspectratio]{wip2.jpg} \\
      \includegraphics[width=0.925\linewidth, keepaspectratio]{wip3.jpg}
    \end{figure}

  }
\end{frame}

%%%%%%%%% Slide 26 %%%%%%%%%%
\begin{frame}[fragile]
  \frametitle{\normalsize{\textbf{
    Deforestation in Amazonia
  }}} 

  \scriptsize{  

    \begin{figure}[h!]
      \centering
      \includegraphics[width=\linewidth, keepaspectratio]{wip4.jpg}
    \end{figure}

  }
\end{frame}

%%%%%%%%% Slide 27 %%%%%%%%%%
\begin{frame}[fragile]
  \frametitle{\normalsize{\textbf{
    Deforestation in Amazonia
  }}} 

  \scriptsize{  

    \begin{figure}[h!]
      \centering
      \includegraphics[width=\linewidth, keepaspectratio]{wip5.jpg}
    \end{figure}

  }
\end{frame}

%%%%%%%%%%%%%%%%%%%%%%%%%%%%%%%%%%%%%%%%%%%%%%%%%%%%%%%%%%%%%%%%%%%%%%%%%%%%%%%
\section{References}
%%%%%%%%%%%%%%%%%%%%%%%%%%%%%%%%%%%%%%%%%%%%%%%%%%%%%%%%%%%%%%%%%%%%%%%%%%%%%%%

\begin{thebibliography}{99}
  
  %%%%%%%%%% Slide 28 %%%%%%%%%%%
  \begin{frame}
    \frametitle{\normalsize{\textbf{
          Bibliography 
    }}}

    \scriptsize{
      
    \bibitem{}
      Cortés, J., Mahecha, M. D., Reichstein, M., and Brenning, A. (2020). 
      Accounting for multiple testing in the analysis of spatio-temporal 
      environmental data. Environmental and Ecological Statistics.

    \bibitem{}
      Pinzon, J. and Tucker, C. (2014). A non-stationary 1981–2012 avhrr ndvi3g
      time series. Remote Sensing, 6(8):6929–6960. 

    } 

  \end{frame}
\end{thebibliography}

%%%%%%%%%% Slide 29 %%%%%%%%%%
\begin{frame}
  \frametitle{\normalsize{\textbf{The End}}}

  \section{}
  \begin{center}
    \huge{Thank You !}
  \end{center}

\end{frame}

%%%%%%%%%%%%%%%%%%%%%%%%%%%%%%%%%%%%%%%%%%%%%%%%%%%%%%%%%%%%%%%%%%%%%%%%%%%%%%%%
\end{document}